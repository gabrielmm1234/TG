	Esse capítulo tem como objetivo, descrever todo o processo de desenvolvimento realizado para detectar possíveis fraudes na CEAP. Inicialmente descrevo em detalhes como foi desenvolvido um ETL para extrair os dados da CEAP, fornecidos pela câmara dos deputados, e carregá-los no OrientDB, além de explicar a modelagem feita para o armazenamento dos dados. Em seguida, explico o funcionamento das consultas realizadas para a detecção de fraudes. Finalmente, apresento a arquitetura do sistema web, e como o sistema se comunica com o OrientDB e apresenta informações relevantes sobre os dados.

\section{ETL e modelo do banco de dados}

\subsection{Dados Abertos}

Foram utilizadas duas bases de dados abertos para o desenvolvimento deste trabalho. A primeira é referente aos dados da cota para exercício da atividade parlamentar, e pode ser obtida no seguinte site da Câmara dos Deputados\footnote{http://www2.camara.leg.br/transparencia/cota-para-exercicio-da-atividade-parlamentar/dados-abertos-cota-parlamentar}. A segunda base diz respeito as doações que cada deputado recebeu de empresas ou pessoas físicas, para a campanha eleitoral de 2014, e pode ser obtida no seguinte site do Tribunal Superior Eleitoral \footnote{http://www.tse.jus.br/eleitor-e-eleicoes/estatisticas/repositorio-de-dados-eleitorais-1/repositorio-de-dados-eleitorais}.

A iniciativa de disponibilizar dados e informações por meio de portais tem como um dos objetivos melhorar a confiança da população nos serviços prestados pelo governo. A transparência governamental é , portanto, uma ótima iniciativa e extremamente benéfica para a população. Porém ainda existem diversos pontos a serem melhorados para que estudos sejam feitos de forma mais rápida e precisa. Um dos maiores desafios desse projeto foi trabalhar com as bases de dados abertas mencionadas nessa seção.

Trabalhar com esses dados abertos se tornou um desafio, porque, cada instituição fornece os dados da própria maneira, essa falta de padronização dificultou bastante a fase de carregamento dos dados. Isso se deve ao fato de que a base de dados do TSE fornece o cpf de cada candidato, já a base da CEAP não fornece um identificador único para o deputado. Portanto, no momento de realizar o cruzamento não existia um identificador único em ambas as bases para fazer o relacionamento entre as bases de dados. Para resolver esse problema, se fez necessário realizar o cruzamento entre as bases por meio do nome de cada candidato, o que também foi um desafio, já que cada base utiliza um nome diferente para cada candidato. Os detalhes do processo de extração, transformação e carregamento serão fornecidos na Seção \ref{etl-subsection}, o ponto principal a ser levantado é que trabalhar com dados abertos no Brasil, para realizar estudos e pesquisas, pode se tornar um desafio devido a falta de padronização entre as bases de dados. 

\subsection{Modelo de dados}

A modelagem dos dados seguiu o modelo GRAPHED \cite{graphed}. O trabalho em questão, busca propor formas de modelagem dos dados para bancos de dados orientado a grafos, uma área já desenvolvida no universo dos SGBD relacionais, mas ainda em evolução na categoria de SGBD orientado a grafos.

A modelagem tem como objetivo dar uma visão geral de como os dados estão organizados no banco de dados, suas propriedades e relacionamentos. Dessa forma, o modelo de dados desenvolvido busca evidenciar as propriedades dos parlamentares, tais como: nome, partido e unidade federativa. Além disso apresenta características que identificam uma empresa como nome e CNPJ, e características atreladas a transação que o deputado faz com uma empresa como o valor e descrição da transação. A Figura \ref{fig:modeloDeDados} apresenta a modelagem  desenvolvida.

\begin{figure}[H]
\centering
\includegraphics[width=.85\textwidth]{CEAP-v3.png}
\caption{Modelo de dados seguindo o formato GRAPHED}
\label{fig:modeloDeDados}
\end{figure}

Como podemos observar na Figura \ref{fig:modeloDeDados}, existem quatro classes que representam as instâncias dos vértices no banco de dados, que são: Parlamentar, Transação, Empresa fornecedora e Pessoa. As setas que saem de uma classe a outra, representa um relacionamento entre essas classes, que no grafo será representado por uma aresta. Dessa forma, uma instância de um parlamentar, realiza transações, que por sua vez é fornecida por uma empresa fornecedora. Esse caminho descrito, representa as transações da CEAP, fornecida pela Câmara dos Deputados.

De forma análoga, uma empresa fornecedora realiza transações, que por sua vez favorece um certo parlamentar. Esse caminho representa os dados das doações das empresas para os deputados nas eleições de 2014, fornecido pelo TSE. 

Os demais relacionamentos, como "socio-de" e "parente-de" tem por objetivo identificar possíveis fraudes na CEAP, uma vez que um parlamentar não pode utilizar a verba da CEAP com serviços de uma empresa que é sócio. Esses dois caminhos, se apresentaram como o maior desafio para o desenvolvimento do trabalho, uma vez que dados de parentesco dos parlamentares não são dados abertos.

\subsection{ETL} \label{etl-subsection}

	O termo ETL vem do inglês \textit{Extract Transform Load}, e corresponde a \textit{softwares} que tem como função:
\begin{itemize}
		\item Extrair os dados de uma determinada fonte.
		\item Transformar os dados de forma que atendam os requisitos da aplicação.
		\item Carregar os dados no banco de dados.
\end{itemize}
	
	A câmara dos deputados fornece todos os dados referentes a CEAP através desse link \href{http://www2.camara.leg.br/transparencia/cota-para-exercicio-da-atividade-parlamentar/dados-abertos-cota-parlamentar}{dados CEAP}. Os dados estão presentes em diferentes formatos: \textit{XML}, \textit{JSON}, \textit{CSV} e \textit{XLSX}. Nesse trabalho foi utilizado o formato \textit{CSV} do ano de 2017. Nesse arquivo \textit{CSV}, os dados estão organizados da seguinte forma:

\begin{table}[H]
\centering
\begin{tabular}{|l|l|l|l|l}
\cline{1-4}
txNomeParlamentar                                 & idecadastro                                       & nuCarteiraParlamentar                             & ... &  \\ \cline{1-4}
ABEL MESQUITA JR.                                 & 178957                                            & 1                                                 & ... &  \\ \cline{1-4}
\begin{tabular}[c]{@{}l@{}}.\\ .\\ .\end{tabular} & \begin{tabular}[c]{@{}l@{}}.\\ .\\ .\end{tabular} & \begin{tabular}[c]{@{}l@{}}.\\ .\\ .\end{tabular} & ... &  \\ \cline{1-4}
ADAIL CARNEIRO                                    & 178864                                            & 92                                                & ... &  \\ \cline{1-4}
\end{tabular}
\caption{Organização dos dados da CEAP}
\label{CEAP-ORG}
\end{table}

	Os dados estão dispostos em linhas e colunas, cada linha representa uma transação diferente para um certo parlamentar. Dentre as colunas disponibilizadas, foram utilizadas as seguintes informações para a fase de extração: TxtDescricao, TxtDescricaoEspecificacao, VlrDocumento, TxNomeParlamentar, IdeCadastro, NumCarteiraParlamentar, SgUF, SgPartido, TxtFornecedor e TxtCNPJCPF.
	
	A partir desses dados, foi desenvolvido um ETL utilizando a linguagem Java, que na fase de extração, lê esses atributos citados acima, linha a linha e instancia um objeto que representa uma transação de um parlamentar com todas as suas informações. Não foi feita nenhuma transformação nos dados antes de carregar para o banco de dados. Por fim, na fase de carregamento, só se fez necessário verificar antes de adicionar um parlamentar ou uma empresa fornecedora, se esse parlamentar ou empresa já estão persistidos; Se estiver persistido, não é criada uma nova instância no banco de dados, e uma referência para o vértice é retornada para que os relacionamentos sejam persistidos em cima dessas refêrencias.

\section{Consultas para detecção de fraudes}

\section{Sistema Web}

\section{Conclusão}