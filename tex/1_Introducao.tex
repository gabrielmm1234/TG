\label{chap:1}

	A política de transparência no Brasil surgiu há alguns anos em nossa sociedade, e tem como principal objetivo auxiliar na confiança da população sobre os serviços prestados pelo governo. O Brasil vive nesse momento um grande caso de corrupção, que é a operação lava jato, e políticas de transparência também tem suas vertentes no combate a corrupção como é apresentado no trabalho feito por Bruna, Cappelli e Claudia \cite{diirrcombate}. O estudo feito por Abramo \cite{abramo2000relaccoes} compara as relações entre índices de percepção de corrupção
e outros indicadores de alguns países latino americanos, e o Brasil se encontra na quadragésima nona posição em um ranking de corrupção dentre 90 países. Em um estudo feito por Filgueiras \cite{filgueiras2009tolerancia}, é feita uma pesquisa de opinião, na qual a Câmara dos Vereadores e a Câmara dos Deputados são as instituições com maior percepção de corrupção no Brasil.

	Dessa forma, o objetivo geral deste trabalho é utilizar a tecnologia de banco de dados orientado a grafos, para auxiliar na detecção de possíveis fraudes em um determinado conjunto de dados. Como foi mencionado no estudo acima, a Câmara dos Deputados está entre as instituições com maior percepção de corrupção, portanto, o conjunto de dados utilizado neste trabalho é a Cota para o Exercício da Atividade Parlamentar – CEAP (antiga verba indenizatória), que é uma cota única mensal destinada a custear os gastos dos deputados exclusivamente vinculados ao exercício da atividade parlamentar.
	
	O \href{http://www2.camara.leg.br/legin/int/atomes/2009/atodamesa-43-21-maio-2009-588364-norma-cd-mesa.html}{Ato da Mesa nº 43 de 2009}, detalha as regras para o uso da CEAP, entretanto um deputado pode realizar algumas transações que não são observadas facilmente pelos responsáveis em fiscalizar essas transações. Por exemplo, o artigo 4, parágrafo 13 do Ato da Mesa nº 43 de 2009, diz: \textit{"Não se admitirá a utilização da Cota para ressarcimento de despesas relativas  a bens fornecidos ou serviços prestados por empresa ou entidade da qual o proprietário ou detentor de qualquer participação seja o Deputado ou parente seu até terceiro grau"}. Dessa forma, o Deputado pode realizar transações que violam essa regra, sendo inviável verificar as relações de parentesco de cada Deputado em cada transação, especialmente se utilizarem tecnologias inadequadas.
	
	Portanto, justifica-se o uso de um banco de dados orientado a grafo para identificar os relacionamentos envolvendo cada transação de um Deputado. Um banco de dados relacional também consegue resolver esse problema, entretanto, com um custo e complexidade bem maior em relação a um banco de dados orientado a grafo. Isso se deve porque os relacionamentos são evidenciados na estrutura de um grafo de forma muito mais natural e simples, onde cada entidade é representada como um nó do grafo e se relaciona com outras entidades por meio de arestas. Devido a essas particularidades, os bancos de dados em grafo vem ganhando bastante popularidade, registrando a maior taxa de mudança de popularidade de 2013 até 2017 \cite{Dbmspopularity}.
	
\section{Objetivos}
	O objetivo geral deste trabalho é o desenvolvimento de um sistema web que fornece informações sobre as trasações de 2017 da CEAP dos deputados de Minas Gerais e Distrito Federal, utilizando o NoSQL baseado em grafos OrientDB para armazenar os dados, e para evidenciar relacionamentos nas transações dos Deputados que violam o artigo 4, parágrafo 13 do Ato da Mesa nº 43 de 2009, que regula a CEAP. 
	
	Os objetivos específicos deste trabalho são:
	\begin{itemize}
		\item Implementar o banco de dados em grafo com os dados abertos da CEAP.
		\item Executar consultas que apresentam informações gerais das transações da CEAP.
		\item Executar consultas que evidenciem relacionamentos fraudulentos.				
		\item Desenvolver um sistema web para apresentar informações a respeito dos dados.
	\end{itemize}

\section{Metodologia}

	Este trabalho foi dividido em duas partes, a primeira teórica e a segunda prática. Na parte teórica foi realizado um estudo baseado em livros, artigos e páginas da \textit{web} sobre os assuntos relacionados a banco de dados, SGBD orientado a grafo, NoSQL, transparência de dados governamentais para a sociedade e etc. Já na parte prática, após obter os dados abertos da CEAP no site da Câmara dos Deputados, foi desenvolvido um ETL para ler os arquivos que contém os dados das transações e popular o banco de dados, em seguida foram realizadas consultas que buscam evidenciar os relacionamentos atrelados a cada transação, posteriormente foi desenvolvido um sistema web de forma que forneça as informações claramente a todos. Por fim, foi realizado uma análise dos resultados obtidos e conclusões finais.

\section{Estrutura do Trabalho}
	Este trabalho está dividido nos seguintes capítulos:
	
	\begin{itemize}
		\item Capítulo 2: são introduzidos os conceitos relacionados a grafos, necessários para a compreensão de um SGBD orientado a grafos. É fornecida uma visão geral de SGBD NoSQL, apontando suas principais características e utilidades. Finalmente explica-se as características de um SGBD orientado a grafo e o estilo arquitetural \textit{REST}.
		\item Capítulo 3: nesse capítulo apresenta-se os meios utilizados para resolver o problema, e uma explicação mais detalhada de todo o processo de desenvolvimento e como as tecnologias auxiliaram nessa resolução.
		\item Capítulo 4: é apresentado a análise dos resultados obtidos ao realizar as consultas e do sistema desenvolvido para fornecer as informações. Essa análise abrange tanto os resultados das consultas que buscam identificar uma transação ilícita, quanto as consultas que apresentam informações gerais sobre a CEAP.
		\item Capítulo 5: finalmente, apresenta-se as conclusões do trabalho realizado e sugestões para trabalhos futuros relacionados com a área.
	\end{itemize}