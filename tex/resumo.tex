	Este trabalho propõe o uso da tecnologia de banco de dados orientado a grafos para a detecção de fraudes na Cota para o Exercício da Atividade Parlamentar – CEAP. Além do uso do banco de dados orientado a grafos, foi desenvolvida uma plataforma web para expor informações importantes a população. O uso dessas tecnologias facilita a manipulação de dados bastante relacionados entre si, tanto em questão de complexidade na consulta, quanto em relação à visualização da informação. A proposta em questão foi validada com um estudo de caso, utilizando os dados abertos da Cota para o Exercício da Atividade Parlamentar da Câmara dos Deputados, para os deputados de Minas Gerais e do Distrito Federal. Foi desenvolvido um ETL para extrair os dados e popular o banco, em seguida as consultas foram realizadas para detectar as fraudes e obter informações a respeito dos dados, finalmente, foi desenvolvido um sistema web que se comunica via REST com o banco de dados para expor as informações a população de forma mais clara e simples.