	Este trabalho propõe o uso da tecnologia de banco de dados orientado a grafos para a detecção de fraudes na Cota para o Exercício da Atividade Parlamentar – CEAP. Além do uso do banco de dados orientado a grafos, será feita uma plataforma web para expor informações importantes a população e um estudo sobre o impacto dessas informações na política brasileira. O uso dessas tecnologias facilita muito a manipulação de dados bastante relacionados entre si, tanto em questão de complexidade na consulta, quanto em relação a visualização da informação. A proposta em questão foi validada com um estudo de caso, utilizando os dados abertos da Cota para o Exercício da Atividade Parlamentar da câmara dos deputados. Foi desenvolvido um ETL para extrair os dados e popular o banco, em seguida as consultas foram realizadas para detectar as fraudes e obter informações a respeito dos dados, finalmente foi desenvolvido um sistema web que se comunica via REST com o banco de dados para expor as informações a população de forma mais clara e simples. Para trabalhos futuros, seria interessante o uso de aprendizagem de máquina para obter mais informações valiosas sobre a CEAP.