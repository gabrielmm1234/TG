Nesta monografia foi desenvolvido um estudo sobre o uso de banco de dados orientado a grafos na detecção de fraudes na cota para exercíco da atividade parlamentar (CEAP). Para isso, após a obtenção dos dados da CEAP do ano de 2017, foi feita uma modelagem seguindo o modelo GRAPHED, para em seguida desenvolver um ETL que armazenasse os dados no OrientDB. Tudo isso possibilitou que fossem feitas consultas que tentassem identificar padrões fraudulentos no conjunto de dados. Além de identificar possíveis fraudes, foi feito um cruzamento com os dados do TSE acerca de doações para a campanha dos deputados nas eleições de 2014, e desenvolvido um sistema web para apresentar informações relevantes acerca dos dados da CEAP e do TSE. A plataforma também tem caráter colaborativo, uma vez que, os dados de parentescos dos Deputados não são abertos e de fácil acesso. Dessa forma, a população pode contribuir com dados e melhorar as chances de detecção de fraudes do sistema.

Os resultados obtidos demonstram que um SGBD orientado a grafos é uma boa solução para manipular dados bastante relacionados entre si. Além disso, ao manipular dados com essa característica de serem bastante relacionados, a estrutura de grafo é uma boa alternativa em relação à estruturas tabulares, pois facilita a identificação dos vínculos entre cada entidade envolvida de forma mais clara e rápida. Seguindo a metodologia proposta foi possível encontrar no cruzamento dos dados da CEAP e do TSE Deputados que utilizaram a CEAP com serviços de empresas que fizeram doações para suas campanhas nas eleições de 2014. Vale lembrar, que isso não caracteriza fraude de acordo com o regimento da CEAP, mas serve de validação do uso da estrutura de grafo no cruzamento entre bases de dados distintas. Por fim, as consultas apresentadas conseguiram detectar padrões fraudulentos utilizando dados fictícios, e ao serem integradas ao sistema colaborativo, cria-se uma boa ferramenta tanto para detecção de fraudes quanto para a transparência da CEAP para a população brasileira.

As vantagens no uso do OrientDB nos dados da CEAP incluem a facilidade de construir as consultas, uma vez que, o OrientDB fornece a flexibilidade de um SGBD NoSQL junto com uma linguagem de consulta derivada da linguagem SQL, utilizada em SGBD relacionais. Além disso, as características de orientação a objetos presentes nesse SGBD facilitam bastante na modelagem, importação e consulta dos dados. Além disso, o suporte a comunicação utilizando o protocolo HTTP, e respostas utilizando o formato JSON, facilitaram bastante na integração do OrientDB com o sistema web desenvolvido.

Por fim, como resultado acadêmico, foi desenvolvido um artigo baseado nesta monografia que será apresentado no Workshop de Transparência em Sistemas - WTrans, no XXXVIII Congresso da Sociedade Brasileira de Computação.

	Seguem algumas propostas de melhorias e trabalhos futuros:
	
	\begin{itemize}
		\item Utilizar técnicas de aprendizagem de máquina e inteligência artificial para encontrar novas informações sobre a CEAP;
		\item Estudar formas de garantir que as informações fornecidas estão corretas, para não realizar alguma injustiça com algum deputado. Uma vez que, durante o processo de leitura, processamento e carregamento dos dados pode ocorrer algum erro;
		\item Expandir o estudo para todos os estados brasileiros e todos os anos disponíveis, e validar o comportamento da estrutura de grafos na presença de um grande volume de dados.
	\end{itemize}