	This work proposes the use of graph oriented databases to detect frauds in Quota for the Exercise of Parliamentary Activity. In addition to the use of the graph oriented databases, a web platform will be developed to expose important information to the population. The use of these technologies, facilitates the manipulation of closely related data, both in terms of query complexity, and information visualization. The proposal in question was validated with a case study, using the open data of the Quota for the Exercise of the Parliamentary Activity of the Chamber of Deputies, for the deputies from Minas Gerais and Distrito Federal. It was developed an ETL to extract the data and fill the database, then the queries were made to detect the fraud and to obtain information about the data, finally a web system was developed that communicates via REST with the database to expose the information to the population more clearly and simply.